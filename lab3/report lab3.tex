\documentclass{article}
\usepackage[utf8]{inputenc}
\usepackage{graphicx}
\usepackage{listings}
\usepackage{color}

\definecolor{dkgreen}{rgb}{0,0.6,0}
\definecolor{gray}{rgb}{0.5,0.5,0.5}
\definecolor{mauve}{rgb}{0.58,0,0.82}

\lstset{language=C,
	numbers=left,
	stepnumber=1,    
	firstnumber=1,
	numberfirstline=true
	aboveskip=5mm,
	belowskip=5mm,
	showstringspaces=false,
	columns=flexible,
	basicstyle={\small\ttfamily},
	numberstyle=\tiny\color{gray},
	keywordstyle=\color{blue},
	commentstyle=\color{dkgreen},
	stringstyle=\color{mauve},
	breaklines=true,
	breakatwhitespace=true
	tabsize=3
}

\title{Artificial Intelligence 1 \\ Lab 2}%Update the lab (assignment number)
\author{Roeland Lindhout (s2954524) \& Younes Moustaghfir (s2909758) \\ AI2/AI3} %Change the names and fill in the student numbers and the group name (AI1/AI2/CS1 etc)
\date{\today}%Update the date

\begin{document}
	
	\maketitle
	
	\section{Constraint Satisfaction Problems}
	
	\subsection*{Solving a small set of equations}
	
	The program took quite some time, and states, to calculate the variables. The only solution that the program found, was A = 3, B = 7 and C =2. This was found in 63254.
	
	\subsection*{Market}
	
	We used three variables, each with a domain from 0 to 100. The three variables represented the number of fruits. Added to each other, they had to equal 100. The second constraint considered the prices: $A*88 + B*99 + C*102 = 10000$. Wherein A represented the oranges, B the grapefruit and C the melons. The reason that they have to be equal to 10000, is that the prices are given in cents. 
	
	There were 5 solutions, in 10308 states. We checked the solutions, and they were all correct. The solutions were as follows: 
	\begin{lstlisting}	
	### Solution 1 ###
	A = 	1 
	B = 	62 
	C = 	37 

	### Solution 2 ###
	A = 	4 
	B = 	48 
	C = 	48 

	### Solution 3 ###
	A = 	7 
	B = 	34 
	C = 	59 

	### Solution 4 ###
	A = 	10 
	B = 	20 
	C = 	70 

	### Solution 5 ###
	A = 	13 
	B = 	6 
	C = 	81 
	\end{lstlisting}
	
	
	\subsection*{Chain of trivial equations}
	
	Without an extra constraint, many states were visited, and 100 solution werer found. The solutions were of course all the integers from 0 up to and including 99. With the constraint of A being 42, 2502 states were visited, and only 1 solution was found, namely all are 42. With the constraint of Z being 42, only 27 states were visited. The reason that this is so much lower, is that if $A = 42$, you have yet to determine all other variables, so you have to test all of them. When $Z = 42$, you can backtrack using the known final value, because each variable gets the value of the next variable. 
	
	\subsection*{Constraint graph}
	
	For this problem, there are 5 variables, namely A to E. Each of these variables has a domain from 1 up to and including 4. We had to introduce an extra variable, to be able to represent the constraint $ C \neq D + 1 $. This variable ranged from 2 to 4, because it is always bigger than 1, since 0 was not in the original domain. 5 (i.e. 4 + 1) also didn't have to be included, because it is impossible for C to be 5, so when D = 4, $ C \neq D + 1 $ is always true.
	
	The problem was solved in 29 states, and 2 solutions were found: 
	\begin{lstlisting}
	### Solution 1 ###
	A = 	3 
	B = 	3 
	C = 	4 
	D = 	2 
	E = 	1 
	F = 	3 

	### Solution 2 ###
	A = 	4 
	B = 	4 
	C = 	2 
	D = 	3 
	E = 	1 
	F = 	4 
\end{lstlisting}


	\subsection*{Cryptarithmetic puzzles}
	
	For the SEND+MORE=MONEY-problem, we had to define 7 variables with a domain ranging from 0 to 9 and 4 variables ranging from 0 to 1. The first 7 variables were all the letters of those words. In this case that were: S, E, N, D, M, O, R, Y. SEND and MORE are like a sum, so carry-over variables are needed. In this case we named them: X1 to X4, because the total length of the words SEND and MORE is 4, therefore there are 4 places where a carry-over could take place. The constraints we chose were: 
	\begin{lstlisting}
	alldiff(S,E,N,D,M,O,R,Y);
	D + E = Y + 10 *X1;
	X1 + N + R = E + 10 * X2;
	X2 + E + O = N + 10 * X3;
	X3 + S + M = O + 10 * X4;
	X4 = M;
	M = 1;
	\end{lstlisting}
Most of the constraints are quite logical, but we chose the value 1 for M, because the values of S and M cannot be higher than 20(because they are both maximally 9). Therefore M cannot be higher than 1. M can also not be 0, because numbers do not start with a zero.
	
	For the UN+UN+NEUF = ONZE-problem, we had to make a small adjustment compared to the code for the first problem described in this subsection, because we did not have two, but three words, so it became possible that the carry-over could also be 2. As a heuristic, we figured that X3, which was the last carry-over, could not be bigger than 1, because at that point, we no longer had 3 letters. We did not need a fourth carry-over, because there is no further addition in the final step. There is only one solution. Namely:
	\begin{lstlisting}
	### Solution 1 ###
	U = 	8 
	N = 	1 
	E = 	9 
	F = 	7 
	O = 	2 
	Z = 	4 
	X1 = 	0 
	X2 = 	2 
	X3 = 	1 
	\end{lstlisting}
	\section{Source code}

	
	

		
\end{document}